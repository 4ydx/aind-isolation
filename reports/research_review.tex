\documentclass[10pt, a4paper]{article}
\usepackage[utf8]{inputenc}

\title{Game Tree Searching by Min / Max Approximation In Review}
\author{Nathan Findley}
\date{March 2017}

\begin{document}

\maketitle

\begin{abstract}

A new game tree search technique called Min / Max Approximation is introduced
by Ronald L. Rivest in 1986.  An iterative approach, it is used to determine
which leaf should be futher expanded so that the root will have a value that
approaches the true values of a fully expanded tree.  When evalutated in 
comparison with minmax alpha-beta pruning
using iterative deepening over the course of nearly 1,000 trials, Min / Max
approximation can be seen to be either a strong or slightly weaker contender 
depending on what kind of resource is limited when evaluating the tree.

\end{abstract}

\textbf{Goals} 

When designing an algorithm to handle all possible branches of all possible
plays within any given game, the immediate limiting factor is time: a perfect
view of the entire tree of gameplay for a particular game is not feasible.

Min / Max Approximation is attempting to solve this problem.  Unlike minimax
alpha-beta pruning with iterative deepening, Min / Max approximation follows a
given branch in a tree because it appears to be beneficial.  As such different
levels of depth are reached in different trees of gameplay.  Effectively all
tips of the tree are evaluated while time permits.  If a tip's value in
combination with all of its parent values still makes it the most interesting
line of play, that particular tip is continuously explored. As to which tip should 
be explored, Min / Max approximation is presented as a possible answer.

Penalties are introduced in order to determine which tip to pursue. This
penalization incorporates depth as well as the approximated value of worthiness
of a particular branch.  Ultimately the overall penalty for a given tip is the
sum of all penalties tracing up to the original root move.   This calculation of
a tip is what distinguishes this algorithm from minimax alpha-beta: the tip that
has the least penalty is the one that is further explored. As soon as a tip's
weight drops its penalty score below any other tip in the partially expanded
tree, that more favorable tip is then explored. One of Min / Max approximations 
strengths is to 
encourage branches that have secondary lines of play that might also be
winnable since the penalty calculation is determined cumulatively rather than
just using a child with the highest value.

Connect four was used to explore whether or not this is a viable algorithm.

Overall Min / Max Approximation appears more effective than traditional 
minimax alpha-beta pruning with iterative deepening as long as the number of moves 
is the limiting resource when comparing the two methods.  According to the original 
publication it remains to be seen if this method can be incorporated into other 
methods in order to produce more efficient algorithms as well as to explore possibilities in
parallelization.

\end{document}
